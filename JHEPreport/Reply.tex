\documentclass[12pt]{article}
\usepackage[letterpaper,margin=0.7in]{geometry}

\usepackage{feynmp}
\usepackage{url}
\usepackage{latexsym}
\usepackage{amsmath}
\usepackage{booktabs}
\usepackage{cite}
\usepackage{cancel}
\usepackage{upgreek}
\usepackage{amssymb}
\usepackage{ stmaryrd }

%\usepackage{palatino,mathpazo} or \usepackage{times}
%\usepackage{textcomp}
%\usepackage{tgpagella}
%\pagestyle{empty}

\newcommand{\QED}{\ensuremath{\Box}}
\newcommand{\mc}{\mathcal}
% \cong{x,y,z} gives ``x == y (mod z)''
\newcommand{\congr}[3]{\ensuremath{#1 \equiv #2 \mod{#3}}}
\newcommand{\bsl}{\char`\\}
\newcommand{\texcmd}[1]{\texttt{\char`\\#1}} 


\begin{document}

\unitlength = 1mm
\title{Reply to the Report on JHEP\_107P\_0621}
\author{A. Buckley, X. Chen, J. Cruz-Martinez, S. Ferrario Ravasio,\\ T. Gehrmann, E. W. N. Glover, S. H{\"o}che, A. Huss, J. Huston,\\ J. M. Lindert, S. Pl{\"a}tzer, M. Sch{\"o}nherr}
\date{}
\maketitle

We would like to thank the referee for his/her detailed reading of the manuscript and constructive criticism. Incorporating the raised points helped to significantly improve the quality of the paper.
In the following we address each individual point raised by the referee. We hope the paper can now be accepted for publication.

%%%%%%%%%%%%%%%%%%%%%%%%%%%
\section*{Everywhere}
\begin{enumerate}
	\item Several plots mention “Les Houches 2019”. Does it mean that some of these results have been already presented in the report of Les Houches 2019? If yes, this should be made clearer (probably in the introduction).

{\bf Update:} This study was initiated during the Les Houches 2019 workshop. Only the fixed order results for Higgs pT distribution were presented in Les Houches 2019 report. We now remove the label regarding "Les Houches 2019" in all plots of this study and add a brief explanation in the second last paragraph of the introduction about the relation to Les Houches 2019 report. 
\end{enumerate}

\section*{Section I}
\begin{enumerate}
\item While not completely overlapping with the present study, some interesting references are missing. For example, arXiv: 1802.09955 and 1805.04446 are complimentary to it and should be cited. In general, I would encourage the authors to present a more detailed review of the theoretical status of VBF at the LHC in the introduction.

{\bf Update:} We rewrite the introduction with more emphasis on the previous studies on VBF, VH and ggF production channels for Higgs-plus-two-jet final states. The two suggested references are also added.

\end{enumerate}
\section*{Section II}
\begin{enumerate}
\item It is correct that PDF4LHC15\_30 NNLO is used for all predictions (even at NLO)? This should be made explicit.

{\bf Update:} Yes, we use the same set of PDFs with the central group values everywhere in this study. This setup helps to quantify the pure different between fixed order and parton shower corrected predictions. This is now explained in this section when introducing the setup of the calculation. 

\item The description of the Powheg predictions around Eq. (5) is a bit unclear and clarifications should be provided. In particular, it is not clear how this prediction differs from a complete NLO QCD+PS prediction. Does it differ by the virtual corrections to the decay only? If possible, an estimate of the approximation would be useful. Also, while Eq. (5) is correct when integrating over the whole phase space, it is not true in the presence of cuts (especially close to resonances). Some comments on it would also be welcome.

{\bf Update:} The text has been modified and it should be more clear now that this corresponds to a complete NLOPS POWHEG prediction. Eq. (5) is valid for the differential (in the underlying Born kinematics) cross section, and it is equivalent to the virtual corrections plus real corrections (integrated over the radiation phase space). 
Since the weight (i.e. the  $\tilde{B}$ function) of events generated according to the POWHEG prescription is identical to the NLO cross section, differential in the underlying Born kinematics (but intergrated over the phase space of additional radiation),  it is sufficient to add this term to the event weight.
The last missing ingredient to have a complete NLOPS prediction in the POWHEG language is to use the real matrix element for $V\to q \bar{q} g$ to generate the hardest emission from the $V$ decay: this is provided directly by the Herwig7 event generator, which contains matrix element corrections for $Z$ and $W$ bosons decay.
No approximation is made since at NLO there is no interference between radiation in production and in decay for this process. 
 
\end{enumerate}

\section*{Section III}
{\bf Update:} I divided Section III into several sections so that it doesn't appear too monolithic. I think it's better. 
\begin{enumerate}
\item ``The gluon-gluon fusion contributions are shown in both the Higgs effective theory (HEFT) and approximate Standard Model, using the reweighting technique described in Sec. II A. " Sec. II A. is not really illuminating on what is actually done there. This should be defined a bit more precisely and in particular how the two predictions differ. Also, why are both computations for ggF not provided in Table 1 (as on Fig.3) for example? Note that the labelling is not consistent between the plots and the table.

{\bf Update:} The reweighting procedure is now specified in Sec. II with notations to distinguish labels of "SM", "HEFT" and "approx SM". {\bf Still need to make clear the ggF lable in Table.1 and Fig.3.}

\item A particular emphasis is put on the distinctions between VBF production with respect to VH. Such a distinction is actually unphysical as both production mechanism are actually undistinguishable (as they interfere) and posses the same physics content. The authors should explain why they want to treat VH as a background as, a priori, no distinction is required.

{\bf Update:} It is indeed not clear what do we mean by VBF or VH production, given that VH production with V decay hadronically shares the same final states with VBF production. We add more explanations now in the introduction along with the text describing Fig. 1.

\item In the footnote of page 2, it is indicated that ttH is not relevant for this study. On the other hand, it is shown on Fig.2 which is a bit contradictory. In particular, the implementation of cuts for ttH is not clear. From the caption, it is says “all hadronic”, probably meaning a 4j+2j\_b final state. In this case, how is implemented the event selection for this process? Are the bottom jets counted as jets? Also, more information on how this process is computed (given it is presented) should be given in the text (and not only in the caption).

{\bf Update:} The ttH process has been included in Fig. 2 for illustration of its impact. However, it is not considered in the rest of the paper. We updated footnote 2 and clarified this. Regarding the implementation of ttH for Fig. 2 we incorporated this information in the main text: for the fiducial selection shown in Fig. 2 (right) indeed the 4j+2j\_b final state has been considered treating bottom jets as jets. References were added for the ATLAS and CMS discovery papers for ttH.

\item How the EW corrections are implemented for each process is unclear. Some references are given in the caption of Fig.2 but it would be more useful to have more details in the text instead.

{\bf Update:} We updated the text accordingly including all relevant references. We also extended Fig. 2 in order to explicitly highlight the impact of the EW corrections.

\item Concerning Fig.2, the event selection should also be mentioned in the text (and not only in the caption).

{\bf Update:} We updated the text accordingly.

\item On page 11, ``The fractional cross section distributions for the Higgs boson production processes are shown in the top panel of Fig. 10 as a function of $\Delta y_{jj}$ using the two leading jets (i.e. the ones with largest pT) for the three different Higgs boson pT cuts." I believe it should be $\Delta y_{jj}^{\text{min}}$ according to what is shown on the plot. This should also be made clearer in the text.

{\bf Update:} It is indeed $\Delta y_{jj}^{\text{min}}$. The text has been corrected, and also a definition of the fractional cross section is provided.

\item Several times, the authors use the word ``original" to describe a jet without defining it explicitely. From the context, one somehow understands that this is a ``Born" jet but such a definition is highly ambiguous in particular at higher order. More information should be provided.

{\bf Update:} The sentence has been reformulated, so that it is more clear that  the original jets correspond to the two final state partons in the Born level configuration. One can imagine to build a mapping from the real to the Born configuration which preserve the t-channel propagators. Since there is only a unique way to do so, the definition of ``Born/original'' jets is unambiguous.

\end{enumerate}
\section*{Section IV}
\begin{enumerate}
\item This section focuses on the study of jet-radius. To my knowledge, such a study has been performed for the first time in arXiv: 1703.05676 (which is not cited). The authors should include this reference and explain how their study goes beyond the known results.

{\bf Update:} Indeed the first study of jet cone size effect was in 1703.05676. We add the reference now in the beginning of Section IV and explain the new contribution from our current study.

\item ``In a previous study, we [...]". I think that ``some of us" would be more appropriate.

{\bf Update:} Yes, this is now corrected.

\end{enumerate}
\section*{Section V}
\begin{enumerate}
\item A table with the cross section should be added as it present more information (and therefore help reproducibility of the work presented). This information should be available to the authors without extra runs.
\item In arXiv: 1803.07943, a comparison of theoretical predictions for VBS is also performed. Given the similarities between VBF and VBS, it would be interesting to qualitatively compare the present findings with the ones presented there. In particular, it seems that the agreement between fixed-order predictions and parton-shower ones is better in VBF than in VBS. Also, the spread of NLO+PS predictions seems to be larger for VBS than for VBF. The authors should comment on this.
\item On the plots, the ``Matching \& PS scheme variation" band should also be provided on the inset showing the ratio to NNLO (for example in Fig. 21, it is not visible in the absolute plots).
\item In addition, the 7-scale variation should also be added on the same inset (probably for the NNLO predictions). It would put the differences observed in perspective.
\item One can see noticeable differences between the various NLO+PS predictions in essentially all the observables. The authors should comments on this and explain how these differences could be explained (if possible).
\item From this study and given that the predictions do not agree very well in some phase-space regions, what are the implications for experimental analysis? Should the experimental measurements should stay away from these regions? Or are these differences negligible given the experimental precision even at high-luminosity LHC?
\item The definition of $d_{23}$ and $d_{34}$ is not given anywhere. This should be corrected.
\item The “Matching \& PS scheme variation” band seems to increase for more exclusive cuts (for example on Fig. 22). Is it true? If yes, the author should comment on it.

{\bf Update:} It is correct. We add explanations in the text when analysising Fig. 22.
\end{enumerate}
\section*{Section VI}
\begin{enumerate}
\item ``Precise theoretical predictions for both signal and background processes are of utmost importance in order to harness the full statistical power of LHC event samples and have therefore been computed up to N3LO accuracy." I believe that this statement is not correct as all pro- cesses meant here are not known up to N3LO (QCD) accuracy. This should be corrected.

{\bf Update:} The comment is correct. Only the inclusive Higgs production in VBF channel has been calculated with N3LO (QCD) accuracy within struture function approach. Background process such as Higgs+2jets production from ggF channel is only available at NLO. We correct the text accordingly. 

\item ``In this manuscript we have presented a detailed comparison of the two types of calculations, i.e. NLOPS and NNLO [..]." I think this is also not correct. To my view, Section V does not qualify (yet?) for a ``detailed" study.

{\bf Update:} We remove the word detailed.


\item ``This paper follows a study with a similar spirit carried out with dijet, Z-boson + jet and Higgs boson + jet production at the LHC." Please provide the references.

{\bf Update:} The reference is now added.
\end{enumerate}



\end{document}
